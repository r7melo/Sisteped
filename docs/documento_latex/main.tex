\documentclass{article}
\usepackage[brazil]{babel}
\usepackage{xcolor}
\usepackage{graphicx}
\usepackage{svg}
\bibliographystyle{acm}
\usepackage{float}
\usepackage{url} 

\title{AULA 01 - LATEX}
\author{Romario Melo}
\date{\year}

\begin{document}




% CRIAÇÃO DA PÁGINA DO DOCUMENTO ====================================
\begin{titlepage}
\centering

\begin{figure}
    \centering
    \includegraphics[width=0.2\linewidth]{imagens/logo_ufc.png}
\end{figure}

{\textbf{UNIVERSIDADE FEDERAL DO CEARÁ \\ CIÊNCIA DA COMPUTAÇÃO}}

\vspace{2.5cm}

{\Large \textbf{PROJETO INTEGRADOR II}} \\
{\large \textbf{Marco 1}}

\vspace{1.5cm}

{\large \textbf{Equipe 2 – Sisteped}}

\vspace{1.5cm}

{\textbf{Participantes:}} \\[0.3cm]
{\normalsize
    Klenisson Mateus Oliveira Ribeiro \\ 
    Kauan Santana de Sousa \\
    José Emanoel Alves Madureira \\
    Francisco Diego Targino Sabino \\
    Antonio Romario Carlos de Melo
}

\vspace{2cm}

{\small Professor Orientador: Bruno Riccelli dos Santos Silva}

\vfill
{CRATEÚS \\ 2025}

\end{titlepage}



% CRIAÇÃO DO SUMÁRIO =================================================
\tableofcontents

% DOCUMENTO ================================
\newpage

%------------------------------------------------------
\section{Introdução}
O \textbf{Sisteped – Sistema de Gestão Pedagógica} é uma ferramenta construída para auxiliar professores no acompanhamento contínuo de alunos e turmas. O sistema centraliza notas, comportamentos, análises e relatórios, fornecendo uma visão clara e personalizada do desenvolvimento escolar, facilitando a tomada de decisões pedagógicas.

Este documento apresenta os requisitos funcionais, não funcionais, diagramas e casos de uso do sistema.

%------------------------------------------------------
\section{Requisitos Funcionais}

A seguir, os requisitos funcionais foram organizados por módulos, conforme a arquitetura proposta.

%------------------------------------------------------
\subsection{Módulo ALU – Alunos}

\subsubsection*{ALU-01 – Cadastro de Alunos}
\begin{itemize}
    \item Adicionar, editar e remover alunos.
    \item Armazenar informações básicas: nome, idade, turma e contato opcional.
\end{itemize}

\subsubsection*{ALU-02 – Histórico do Aluno}
\begin{itemize}
    \item Visualizar histórico consolidado de notas e comportamentos por aluno.
\end{itemize}

\subsubsection*{ALU-03 – Busca e Filtros de Alunos}
\begin{itemize}
    \item Buscar alunos por nome, turma ou tags de comportamento/desempenho.
\end{itemize}

%------------------------------------------------------
\subsection{Módulo TUR – Turmas}

\subsubsection*{TUR-01 – Gerenciamento de Turmas}
\begin{itemize}
    \item Criar turmas personalizadas.
    \item Adicionar ou remover alunos de cada turma.
\end{itemize}

\subsubsection*{TUR-02 – Visão Geral da Turma}
\begin{itemize}
    \item Exibir resumo de desempenho médio da turma.
\end{itemize}

%------------------------------------------------------
\subsection{Módulo NOT – Notas}

\subsubsection*{NOT-01 – Registro de Notas}
\begin{itemize}
    \item Lançar notas individuais.
    \item Adicionar tags de conteúdo ou competência.
    \item Visualizar histórico por aluno ou turma.
\end{itemize}

\subsubsection*{NOT-02 – Média e Estatísticas}
\begin{itemize}
    \item Calcular média por conteúdo, aluno e turma.
\end{itemize}

\subsubsection*{NOT-03 – Notas por Conteúdo}
\begin{itemize}
    \item Filtrar notas por matéria, conteúdo ou competência.
\end{itemize}

%------------------------------------------------------
\subsection{Módulo COM – Comportamentos}

\subsubsection*{COM-01 – Registro de Comportamentos}
\begin{itemize}
    \item Criar cards de comportamento.
    \item Associar comportamentos a alunos.
    \item Registrar histórico individual e por turma.
\end{itemize}

\subsubsection*{COM-02 – Métricas de Frequência}
\begin{itemize}
    \item Identificar padrões de comportamentos positivos e negativos.
\end{itemize}

%------------------------------------------------------
\subsection{Módulo MON – Monitoramento e Análises}

\subsubsection*{MON-01 – Análises Acadêmicas}
\begin{itemize}
    \item Exibir gráficos e indicadores de desempenho.
\end{itemize}

\subsubsection*{MON-02 – Análises Comportamentais}
\begin{itemize}
    \item Análise de frequência por tipo de comportamento.
\end{itemize}

\subsubsection*{MON-03 – Comparação Temporal}
\begin{itemize}
    \item Comparação da evolução dos alunos ao longo do tempo.
\end{itemize}

\subsubsection*{MON-04 – Painel Geral do Professor}
\begin{itemize}
    \item Dashboard consolidado com visão de turmas, notas e alertas.
\end{itemize}

%------------------------------------------------------
\subsection{Módulo REL – Relatórios}

\subsubsection*{REL-01 – Exportação e Relatórios}
\begin{itemize}
    \item Gerar relatórios para impressão ou compartilhamento.
    \item Exportar para CSV ou PDF.
\end{itemize}

\subsubsection*{REL-02 – Relatório Consolidado da Turma}
\begin{itemize}
    \item Documento com médias, estatísticas e análises comportamentais.
\end{itemize}

%------------------------------------------------------
\subsection{Módulo SYS – Sistema / Infraestrutura}

\subsubsection*{SYS-01 – Armazenamento Local}
\begin{itemize}
    \item Banco de dados local com suporte a modo offline.
\end{itemize}

\subsubsection*{SYS-02 – Backup Local Manual}
\begin{itemize}
    \item Exportar e importar backups do banco.
\end{itemize}

\subsubsection*{SYS-03 – Tema Claro/Escuro}
\begin{itemize}
    \item Alternar entre temas para melhor usabilidade.
\end{itemize}

%------------------------------------------------------
\section{Requisitos Não Funcionais}

\subsection*{1. Usabilidade}
\begin{itemize}
    \item Ações principais em até 5 segundos.
    \item Localização de funções em até 3 cliques.
\end{itemize}

\subsection*{2. Performance}
\begin{itemize}
    \item CRUD com resposta inferior a 200ms.
    \item Filtros carregando em até 1s com 300+ alunos.
\end{itemize}

\subsection*{3. Segurança}
\begin{itemize}
    \item Armazenamento com criptografia AES-256.
    \item Backup íntegro em 100\% das tentativas.
\end{itemize}

\subsection*{4. Portabilidade}
\begin{itemize}
    \item Compatível com Windows, Linux e macOS.
    \item Suporte a telas de 768px a 4K.
\end{itemize}

\subsection*{5. Escalabilidade}
\begin{itemize}
    \item Impacto máximo de 10\% ao adicionar novos módulos.
    \item Suporte a até 10 mil registros sem perda notável.
\end{itemize}

%------------------------------------------------------
\newpage
\section{Diagramas}

\subsection{Diagrama de Classes}
\begin{figure}[H]
    \centering
    \includesvg[width=\textwidth]{imagens/Diagram de Classes - Sisteped.svg}
    \caption{Diagrama de Classes do Sisteped}
\end{figure}




\subsection{Casos de Uso}
\begin{figure}[H]
    \centering
    \includegraphics[width=\textwidth]{imagens/use cases.png}
    \caption{Diagrama de Caso de Uso}
    \label{fig:casosdeuso}
\end{figure}



%======================================================
\newpage
\section{Casos de Uso}

Esta seção descreve os principais casos de uso do sistema Sisteped, apresentando atores envolvidos, pré-condições, fluxos, pós-condições e possíveis variações para cada cenário. O objetivo é documentar claramente o comportamento esperado do sistema.

\subsection{Efetuar Login}

\subsubsection*{Atores}
\begin{itemize}
    \item Professor
    \item Administrador
\end{itemize}

\subsubsection*{Pré-condição}
\begin{itemize}
    \item O ator deve possuir um registro válido de email e senha.
\end{itemize}

\subsubsection*{Fluxo Principal}
\begin{enumerate}
    \item O ator acessa a página de login.
    \item O sistema exibe os campos para credenciais.
    \item O ator insere email e senha.
    \item O ator clica em “Entrar”.
    \item O sistema valida as credenciais.
    \item O sistema concede acesso e redireciona para o dashboard.
\end{enumerate}

\subsubsection*{Pós-condição}
\begin{itemize}
    \item O ator está autenticado no sistema.
\end{itemize}

\subsubsection*{Fluxo Alternativo}
\begin{itemize}
    \item Credenciais inválidas resultam em mensagem de erro.
\end{itemize}

%------------------------------------------------------
\subsection{Cadastrar Aluno}

\subsubsection*{Atores}
\begin{itemize}
    \item Professor
\end{itemize}

\subsubsection*{Pré-condição}
\begin{itemize}
    \item O professor deve ter acesso ao menu “Alunos”.
\end{itemize}

\subsubsection*{Fluxo Principal}
\begin{enumerate}
    \item O professor acessa o menu Alunos.
    \item Seleciona “Adicionar Aluno”.
    \item Preenche os dados: nome, idade, turma, contato.
    \item Confirma o cadastro.
    \item O sistema salva o registro no banco local.
\end{enumerate}

\subsubsection*{Pós-condição}
\begin{itemize}
    \item O aluno é cadastrado no sistema.
\end{itemize}

\subsubsection*{Fluxo Alternativo}
\begin{itemize}
    \item O professor pode realizar o backup manual após o cadastro.
\end{itemize}

%------------------------------------------------------
\subsection{Gerenciar Dados Cadastrais da Turma}

\subsubsection*{Atores}
\begin{itemize}
    \item Administrador
\end{itemize}

\subsubsection*{Pré-condição}
\begin{itemize}
    \item Administrador logado.
    \item Classes Turma, Aluno e Professor existentes.
\end{itemize}

\subsubsection*{Fluxo Principal}
\begin{enumerate}
    \item O administrador seleciona a turma desejada.
    \item Adiciona ou remove alunos da turma.
    \item Associa ou remove professores da turma.
    \item O sistema atualiza os relacionamentos.
\end{enumerate}

\subsubsection*{Pós-condição}
\begin{itemize}
    \item A composição da turma é atualizada.
\end{itemize}

\subsubsection*{Fluxo Alternativo}
\begin{itemize}
    \item O sistema impede alterações caso haja pendências (ex.: notas não lançadas).
\end{itemize}

%------------------------------------------------------
\subsection{Registrar e Gerenciar Notas}

\subsubsection*{Atores}
\begin{itemize}
    \item Professor
\end{itemize}

\subsubsection*{Pré-condição}
\begin{itemize}
    \item Professor logado.
    \item Turma, Aluno e Avaliação previamente cadastrados.
\end{itemize}

\subsubsection*{Fluxo Principal}
\begin{enumerate}
    \item O professor seleciona a turma.
    \item Seleciona ou cria a avaliação.
    \item Insere notas (float) para cada aluno.
    \item O sistema armazena as notas.
\end{enumerate}

\subsubsection*{Pós-condição}
\begin{itemize}
    \item As notas da avaliação são registradas.
\end{itemize}

\subsubsection*{Fluxo Alternativo}
\begin{itemize}
    \item Notas inválidas resultam em erro e solicitação de correção.
\end{itemize}

%------------------------------------------------------
\subsection{Registrar Incidente de Comportamento}

\subsubsection*{Atores}
\begin{itemize}
    \item Professor
\end{itemize}

\subsubsection*{Pré-condição}
\begin{itemize}
    \item Professor logado.
    \item Aluno selecionado.
\end{itemize}

\subsubsection*{Fluxo Principal}
\begin{enumerate}
    \item O professor abre a função de Registro de Comportamento.
    \item Seleciona o aluno.
    \item Preenche tag, data e observação.
    \item O sistema cria o registro de comportamento.
\end{enumerate}

\subsubsection*{Pós-condição}
\begin{itemize}
    \item O incidente é armazenado e associado ao aluno.
\end{itemize}

\subsubsection*{Fluxo Alternativo}
\begin{itemize}
    \item Tentativa inválida gera mensagem de erro.
\end{itemize}

%------------------------------------------------------
\subsection{Calcular Médias e Estatísticas}

\subsubsection*{Atores}
\begin{itemize}
    \item Professor
\end{itemize}

\subsubsection*{Pré-condição}
\begin{itemize}
    \item Existem notas registradas.
\end{itemize}

\subsubsection*{Fluxo Principal}
\begin{enumerate}
    \item O professor acessa a tela de estatísticas.
    \item O sistema calcula as médias automaticamente.
    \item O sistema exibe os resultados.
\end{enumerate}

\subsubsection*{Pós-condição}
\begin{itemize}
    \item Estatísticas exibidas com sucesso.
\end{itemize}

\subsubsection*{Fluxo Alternativo}
\begin{itemize}
    \item Caso não existam notas, o sistema exibe mensagem de erro.
\end{itemize}

%------------------------------------------------------
\subsection{Analisar Desempenho e Comportamentos}

\subsubsection*{Atores}
\begin{itemize}
    \item Professor
    \item Coordenador
\end{itemize}

\subsubsection*{Pré-condição}
\begin{itemize}
    \item Usuário acessa o painel de análises.
\end{itemize}

\subsubsection*{Fluxo Principal}
\begin{enumerate}
    \item Usuário seleciona filtros (aluno, turma, conteúdo).
    \item O sistema gera gráficos e indicadores.
    \item Usuário revisa as informações exibidas.
\end{enumerate}

\subsubsection*{Pós-condição}
\begin{itemize}
    \item Indicadores são exibidos.
\end{itemize}

\subsubsection*{Fluxo Alternativo}
\begin{itemize}
    \item Usuário pode adicionar novo comportamento durante a análise.
\end{itemize}

%------------------------------------------------------
\subsection{Consultar Histórico Escolar e Desempenho}

\subsubsection*{Atores}
\begin{itemize}
    \item Aluno
\end{itemize}

\subsubsection*{Pré-condição}
\begin{itemize}
    \item Aluno logado.
\end{itemize}

\subsubsection*{Fluxo Principal}
\begin{enumerate}
    \item O aluno solicita a consulta do histórico.
    \item O sistema recupera registros de Avaliação e Comportamento.
    \item O sistema exibe notas, datas e observações.
\end{enumerate}

\subsubsection*{Pós-condição}
\begin{itemize}
    \item Histórico completo é exibido.
\end{itemize}

\subsubsection*{Fluxo Alternativo}
\begin{itemize}
    \item Se não houver registros, exibir “Nenhum registro encontrado”.
\end{itemize}

%------------------------------------------------------
\subsection{UC09 – Gerar Backup Manual}

\subsubsection*{Atores}
\begin{itemize}
    \item Professor
\end{itemize}

\subsubsection*{Pré-condição}
\begin{itemize}
    \item Sistema possui dados armazenados.
\end{itemize}

\subsubsection*{Fluxo Principal}
\begin{enumerate}
    \item O professor abre as configurações.
    \item Seleciona “Gerar Backup”.
    \item O sistema compacta o banco local.
    \item O usuário salva o arquivo.
\end{enumerate}

\subsubsection*{Pós-condição}
\begin{itemize}
    \item Arquivo de backup gerado.
\end{itemize}

\subsubsection*{Fluxo Alternativo}
\begin{itemize}
    \item Backup sem alterações pode gerar arquivo inconsistente.
\end{itemize}

%======================================================
\newpage
\section{Repositório e Protótipo do Sistema}

Esta seção apresenta os artefatos digitais essenciais para o desenvolvimento do Sisteped, incluindo o repositório oficial do código-fonte e o protótipo elaborado no Figma.

%------------------------------------------------------
\subsection{Repositório no GitHub}

O código-fonte do Sisteped está disponível publicamente no GitHub, permitindo versionamento, colaboração e auditoria do desenvolvimento:

\begin{itemize}
    \item \textbf{Repositório oficial:}\\
    \url{https://github.com/r7melo/Sisteped}
\end{itemize}

O repositório contém:
\begin{itemize}
    \item Estrutura completa do projeto;
    \item Código-fonte de desenvolvimento inicial;
    \item Documentações complementares.
\end{itemize}

%------------------------------------------------------
\subsection{Protótipo}

O protótipo de interface do usuário foi desenvolvido no Figma com o objetivo de validar a experiência de uso e definir os fluxos principais antes da implementação. Ele pode ser acessado por meio do link:

\begin{itemize}
    \item \textbf{Protótipo no Figma:}
    \url{https://www.figma.com/design/WE4tHmzitXWEictT3dCRfe/Sisteped?node-id=0-1\&p=f}
\end{itemize}

O protótipo apresenta as principais telas funcionais do sistema, incluindo:

\begin{itemize}
    \item Tela de Login;
    \item Listagem de Turmas;
    \item Listagem de Alunos;
    \item Listagem de Notas;
    \item Telas de Gráficos de Desempenho;
    \item Tela de Relatórios;
    \item Cadastro de Turmas;
    \item Cadastro de Aluno;
    \item Cadastro de Notas;
    \item Geração de Relatórios.
\end{itemize}

Essas telas representam o fluxo essencial do Sisteped e servem como referência visual para orientar o desenvolvimento da interface final do sistema.




\end{document}

