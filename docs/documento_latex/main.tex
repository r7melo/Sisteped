\documentclass{article}
\usepackage[brazil]{babel}
\usepackage{xcolor}
\usepackage{graphicx}
\usepackage{svg}
\bibliographystyle{acm}
\usepackage{float}
\usepackage{url} 

\title{AULA 01 - LATEX}
\author{Romario Melo}
\date{\year}

\begin{document}




% CRIAÇÃO DA PÁGINA DO DOCUMENTO ====================================
\begin{titlepage}
\centering

\begin{figure}
    \centering
    \includegraphics[width=0.2\linewidth]{imagens/logo_ufc.png}
\end{figure}

{\textbf{UNIVERSIDADE FEDERAL DO CEARÁ \\ CIÊNCIA DA COMPUTAÇÃO}}

\vspace{2.5cm}

{\Large \textbf{PROJETO INTEGRADOR II}} \\
{\large \textbf{Marco 1}}

\vspace{1.5cm}

{\large \textbf{Equipe 2 – Sisteped}}

\vspace{1.5cm}

{\textbf{Participantes:}} \\[0.3cm]
{\normalsize
    Klenisson Mateus Oliveira Ribeiro \\ 
    Kauan Santana de Sousa \\
    José Emanoel Alves Madureira \\
    Francisco Diego Targino Sabino \\
    André Luis Felício da Silva \\ 
    Antonio Romario Carlos de Melo
}

\vspace{2cm}

{\small Professor Orientador: Bruno Riccelli dos Santos Silva}

\vfill
{CRATEÚS \\ 2025}

\end{titlepage}



% CRIAÇÃO DO SUMÁRIO =================================================
\tableofcontents

% DOCUMENTO ================================
\newpage

%------------------------------------------------------
\section{Introdução}
O \textbf{Sisteped – Sistema de Gestão Pedagógica} é uma ferramenta construída para auxiliar professores no acompanhamento contínuo de alunos e turmas. O sistema centraliza notas, comportamentos, análises e relatórios, fornecendo uma visão clara e personalizada do desenvolvimento escolar, facilitando a tomada de decisões pedagógicas.

Este documento apresenta os requisitos funcionais, não funcionais, diagramas e casos de uso do sistema.

%------------------------------------------------------
\section{Requisitos Funcionais}

A seguir, os requisitos funcionais foram organizados por módulos, conforme a arquitetura proposta.

%------------------------------------------------------
\subsection{Módulo ALU – Alunos}

\subsubsection*{ALU-01 – Cadastro de Alunos}
\begin{itemize}
    \item Adicionar, editar e remover alunos.
        \begin{itemize}
            \item Formulário de cadastro com campos do Aluno: 
            nomeCompleto, dataNascimento, filiação, naturalidade, nacionalidade, identidade, CPF.
            \item Permite adicionar dados de Endereço (rua, número, bairro, cidade) e Contato (telefone, email).
            \item Validação em tempo real:
            \begin{itemize}
                \item CPF válido e único.
                \item Data de nascimento coerente (idade mínima 6 anos).
                \item Email com formato correto.
            \end{itemize}
            \item Botões: "Salvar", "Cancelar" e "Limpar campos".
            \item Mensagens de erro exibidas em caso de dados inválidos.
        \end{itemize}
\end{itemize}

\subsubsection*{ALU-02 – Histórico do Aluno}
\begin{itemize}
    \item Visualizar histórico consolidado de notas e comportamentos por aluno.
        \begin{itemize}
            \item Exibe todas as Avaliação do aluno, incluindo conteúdo, nota, data e tipo.
            \item Mostra Comportamentos registrados (tag, data, observação).
            \item Filtros por período, disciplina ou tipo de comportamento.
            \item Exportação em PDF ou CSV.
            \item Indicadores visuais:
            \begin{itemize}
                \item Notas abaixo da média destacadas.
                \item Comportamentos recorrentes sinalizados.
            \end{itemize}
        \end{itemize}
\end{itemize}

\subsubsection*{ALU-03 – Busca e Filtros de Alunos}
\begin{itemize}
    \item Buscar alunos por nome, turma ou tags de comportamento/desempenho.
        \begin{itemize}
            \item Busca instantânea conforme o usuário digita.
            \item Filtros combinados: turma, média de notas, tags de comportamento.
            \item Ordenação por nome, idade, média ou quantidade de incidentes.
            \item Resultados exibidos em tabela, com rolagem infinita.
        \end{itemize}
\end{itemize}

%------------------------------------------------------
\subsection{Módulo TUR – Turmas }

\subsubsection*{TUR-01 – Manutenção de Turmas}
\begin{itemize}
    \item Cadastrar nova turma.
        \begin{itemize}
            \item Campos: nome, anoLetivo.
            \item Validação: impedir duplicidade de nome no mesmo ano letivo.
        \end{itemize}
    \item Editar dados da turma.
    \item Excluir ou inativar turmas.
    \item Listar turmas cadastradas (visão simples de lista).
\end{itemize}

\subsubsection*{TUR-02 – Alocação de Pessoas}
\begin{itemize}
    \item Gerenciar alunos na turma.
        \begin{itemize}
            \item Adicionar aluno à turma (relacionamento N:N).
            \item Remover aluno da turma.
            \item Validação: verificar se o aluno já está matriculado.
        \end{itemize}
    \item Gerenciar professores na turma.
        \begin{itemize}
            \item Vincular professores responsáveis.
            \item Desvincular professor.
        \end{itemize}
\end{itemize}

%------------------------------------------------------
\subsection{Módulo NOT – Notas }

\subsubsection*{NOT-01 – Registro de Avaliações e Notas}
\begin{itemize}
    \item Criar avaliação.
        \begin{itemize}
            \item Campos: descrição, data, tipo.
        \end{itemize}
    \item Lançar notas para os alunos.
        \begin{itemize}
            \item Entrada de valor numérico associado ao Aluno e à Avaliação.
            \item Validação: aceitar apenas valores entre 0 e 10.
        \end{itemize}
    \item Editar nota lançada.
    \item Excluir lançamento de nota.
\end{itemize}

\subsubsection*{NOT-02 – Categorização de Conteúdo}
\begin{itemize}
    \item Gerenciar tags de conteúdo/competência.
        \begin{itemize}
            \item Criar, editar e excluir tags para associar às avaliações (ex: "Matemática", "Física").
        \end{itemize}
    \item Consultar notas lançadas (lista simples filtrada por aluno ou data).
\end{itemize}

%------------------------------------------------------
%------------------------------------------------------
\subsection{Módulo COM – Comportamentos}

\subsubsection*{COM-01 – Registro de Ocorrências}
\begin{itemize}
    \item Registrar novo comportamento.
        \begin{itemize}
            \item Campos: aluno, tag (tipo de comportamento), data, observação textual.
            \item Validação: data não pode ser futura.
        \end{itemize}
        
    \item Editar observação ou data do comportamento.
    \item Excluir registro de comportamento.

    \item Registrar nova tag.
        \begin{itemize}
            \item Pode ser registrado novas tags.
            \item Validação: Evita redundância nas tags.
        \end{itemize}
    \item Renomear tag.
    \item Excluir tag.
\end{itemize}

\subsubsection*{COM-02 – Histórico Individual}
\begin{itemize}
    \item Listar comportamentos de um aluno específico.
        \begin{itemize}
            \item Visualização em lista cronológica simples (data e descrição).
        \end{itemize}
\end{itemize}

%------------------------------------------------------
\subsection{Módulo MON – Monitoramento e Análises}

\subsubsection*{MON-01 – Painel Acadêmico (Notas)}
\begin{itemize}
    \item Visualizar desempenho da Turma.
        \begin{itemize}
            \item Gráfico de média geral da turma por avaliação \textbf{(Gráfico de Colunas)}.
            \item Distribuição de notas: quantos alunos acima/abaixo da média \textbf{(Gráfico de Rosca)}.
        \end{itemize}
    \item Visualizar desempenho do Aluno.
        \begin{itemize}
            \item Gráfico de evolução de notas ao longo do tempo \textbf{(Gráfico de Linha)}.
            \item Comparativo do aluno em relação à média da turma \textbf{(Gráfico Combinado)}.
        \end{itemize}
\end{itemize}

\subsubsection*{MON-02 – Painel Comportamental}
\begin{itemize}
    \item Métricas de frequência.
        \begin{itemize}
            \item Frequência por tipo de tag (ex: 30\% Conversa, 10\% Atraso) \textbf{(Gráfico de Barras Horizontais)}.
            \item Identificação de padrões: dias com mais incidências \textbf{(Histograma Semanal)}.
        \end{itemize}
    \item Alertas e Destaques.
        \begin{itemize}
            \item Dashboard destacando alunos com maior recorrência de comportamentos \textbf{(Tabela com Barras)}.
        \end{itemize}
\end{itemize}

\subsubsection*{MON-03 – Visão Gerencial (Dashboard Professor)}
\begin{itemize}
    \item Visão consolidada (Overview).
        \begin{itemize}
            \item Resumo rápido em cards: "Média da Turma Hoje", "Total de Ocorrências" \textbf{(KPI Cards)}.
        \end{itemize}
    \item Comparação Temporal.
        \begin{itemize}
            \item Gráficos comparando o bimestre atual com o anterior \textbf{(Gráfico de Barras Agrupadas)}.
        \end{itemize}
\end{itemize}

%------------------------------------------------------
\subsection*{Glossário de Visualizações (Especificação Técnica)}

\begin{itemize}
    \item \textbf{Gráfico de Colunas:} Comparação simples. Mostra a nota média de cada prova uma ao lado da outra.
    
    \item \textbf{Gráfico de Rosca:} Mostra a proporção. Ex: Metade da turma tirou nota azul, metade tirou nota vermelha.
    
    \item \textbf{Gráfico de Linha:} Mostra a evolução. Permite ver se as notas do aluno estão subindo ou descendo ao longo do ano.
    
    \item \textbf{Gráfico Combinado:} Junta barras (nota do aluno) e uma linha (média da turma) no mesmo lugar para comparar se o aluno foi melhor ou pior que a sala.
    
    \item \textbf{Gráfico de Barras Horizontais:} Lista os tipos de comportamento (ex: "Conversa"). É horizontal para caber o texto do nome do comportamento.
    
    \item \textbf{Histograma Semanal:} Gráfico simples que mostra qual dia da semana (Seg-Sex) teve mais ocorrências registradas.
    
    \item \textbf{Tabela com Barras:} Uma lista com o nome dos alunos mais indisciplinados, com uma barrinha colorida ao lado para destacar quem tem mais ocorrências.
    
    \item \textbf{KPI Cards:} Cartões com números grandes (ex: "7.5"). Serve para o professor bater o olho e saber a média ou total de faltas do dia.
    
    \item \textbf{Gráfico de Barras Agrupadas:} Coloca duas barras juntas (Bimestre 1 e Bimestre 2) para ver se a turma melhorou ou piorou de um bimestre para o outro.
\end{itemize}

%------------------------------------------------------
\subsection{Módulo REL – Relatórios (Exportação Estática)}

\subsubsection*{REL-01 – Listagens Simples}
\begin{itemize}
    \item Gerar lista de chamada/alunos da turma (PDF/CSV).
    \item Gerar boletim simples (lista de notas por aluno).
    \item Gerar extrato de comportamentos (lista textual de ocorrências para assinatura dos pais).
\end{itemize}


%------------------------------------------------------
\section{Requisitos Não Funcionais}

\subsection*{1. Usabilidade (Experiência do Usuário)}
\begin{itemize}
    \item \textbf{Eficiência de Uso:}
        \begin{itemize}
            \item Ações principais (lançar nota, registrar chamada) devem ser concluídas em até 5 segundos.
            \item Feedback visual imediato (mensagens de sucesso ou erro) para qualquer ação do usuário.
        \end{itemize}
    \item \textbf{Facilidade de Aprendizado:}
        \begin{itemize}
            \item Localização de qualquer função do sistema em até 3 cliques a partir da tela inicial.
            \item Interface limpa, utilizando ícones padrões e textos explicativos (tooltips) nos botões.
        \end{itemize}
\end{itemize}

\subsection*{2. Performance (Desempenho)}
\begin{itemize}
    \item \textbf{Latência de Dados:}
        \begin{itemize}
            \item Operações de CRUD (Cadastro, Leitura, Atualização, Deleção) com tempo de resposta inferior a 200ms.
            \item Otimização de queries SQL para evitar travamentos durante o salvamento.
        \end{itemize}
    \item \textbf{Carregamento de Listagens:}
        \begin{itemize}
            \item Filtros e relatórios carregando em até 1s, mesmo em cenários de carga com 300+ alunos registrados.
            \item Paginação automática nas tabelas para não sobrecarregar a memória.
        \end{itemize}
\end{itemize}

\subsection*{3. Segurança (Proteção de Dados)}
\begin{itemize}
    \item \textbf{Integridade e Sigilo:}
        \begin{itemize}
            \item Armazenamento de dados sensíveis com criptografia padrão AES-256.
            \item Senhas dos professores devem ser armazenadas utilizando Hash seguro (ex: bcrypt ou Argon2), nunca em texto plano.
        \end{itemize}
    \item \textbf{Disponibilidade:}
        \begin{itemize}
            \item Garantia de recuperação: Backup íntegro e funcional em 100\% das tentativas de restauração.
            \item Controle de sessão com logout automático após período de inatividade.
        \end{itemize}
\end{itemize}

\subsection*{4. Portabilidade (Compatibilidade)}
\begin{itemize}
    \item \textbf{Multiplataforma:}
        \begin{itemize}
            \item Compatível nativamente com os sistemas operacionais Windows (10/11), Linux (Distros principais) e macOS.
            \item Instalação simplificada com dependências mínimas empacotadas.
        \end{itemize}
    \item \textbf{Adaptabilidade Visual:}
        \begin{itemize}
            \item Suporte a telas com resolução desde 768px (laptops antigos/tablets) até 4K (monitores modernos).
            \item Layout responsivo que se ajusta ao redimensionamento da janela sem quebrar os componentes.
        \end{itemize}
\end{itemize}

\subsection*{5. Escalabilidade (Crescimento)}
\begin{itemize}
    \item \textbf{Arquitetura Modular:}
        \begin{itemize}
            \item Estrutura de código organizada para garantir um impacto máximo de 10\% na performance ao adicionar novos módulos no futuro.
            \item Baixo acoplamento entre as classes para facilitar manutenção.
        \end{itemize}
    \item \textbf{Volume de Dados:}
        \begin{itemize}
            \item Suporte robusto a até 10 mil registros (histórico de anos) sem perda notável de desempenho nas consultas.
        \end{itemize}
\end{itemize}

%------------------------------------------------------
\newpage
\section{Diagramas}

\subsection{Diagrama de Classes}
\begin{figure}[H]
    \centering
    \includesvg[width=\textwidth]{imagens/Diagram de Classes - Sisteped.svg}
    \caption{Diagrama de Classes do Sisteped}
\end{figure}




\subsection{Casos de Uso}
\begin{figure}[H]
    \centering
    \includegraphics[width=\textwidth]{imagens/use cases.png}
    \caption{Diagrama de Caso de Uso}
    \label{fig:casosdeuso}
\end{figure}



%======================================================
\newpage
\section{Casos de Uso}

Esta seção descreve os principais casos de uso do sistema Sisteped, apresentando atores envolvidos, pré-condições, fluxos, pós-condições e possíveis variações para cada cenário. O objetivo é documentar claramente o comportamento esperado do sistema.

\subsection{Efetuar Login}

\subsubsection*{Atores}
\begin{itemize}
    \item Professor
    \item Administrador
\end{itemize}

\subsubsection*{Pré-condição}
\begin{itemize}
    \item O ator deve possuir um registro válido de email e senha.
\end{itemize}

\subsubsection*{Fluxo Principal}
\begin{enumerate}
    \item O ator acessa a página de login.
    \item O sistema exibe os campos para credenciais.
    \item O ator insere email e senha.
    \item O ator clica em “Entrar”.
    \item O sistema valida as credenciais.
    \item O sistema concede acesso e redireciona para o dashboard.
\end{enumerate}

\subsubsection*{Pós-condição}
\begin{itemize}
    \item O ator está autenticado no sistema.
\end{itemize}

\subsubsection*{Fluxo Alternativo}
\begin{itemize}
    \item Credenciais inválidas resultam em mensagem de erro.
\end{itemize}

%------------------------------------------------------
\subsection{Cadastrar Aluno}

\subsubsection*{Atores}
\begin{itemize}
    \item Professor
\end{itemize}

\subsubsection*{Pré-condição}
\begin{itemize}
    \item O professor deve ter acesso ao menu “Alunos”.
\end{itemize}

\subsubsection*{Fluxo Principal}
\begin{enumerate}
    \item O professor acessa o menu Alunos.
    \item Seleciona “Adicionar Aluno”.
    \item Preenche os dados: nome, idade, turma, contato.
    \item Confirma o cadastro.
    \item O sistema salva o registro no banco local.
\end{enumerate}

\subsubsection*{Pós-condição}
\begin{itemize}
    \item O aluno é cadastrado no sistema.
\end{itemize}

\subsubsection*{Fluxo Alternativo}
\begin{itemize}
    \item O professor pode realizar o backup manual após o cadastro.
\end{itemize}

%------------------------------------------------------
\subsection{Gerenciar Dados Cadastrais da Turma}

\subsubsection*{Atores}
\begin{itemize}
    \item Administrador
\end{itemize}

\subsubsection*{Pré-condição}
\begin{itemize}
    \item Administrador logado.
    \item Classes Turma, Aluno e Professor existentes.
\end{itemize}

\subsubsection*{Fluxo Principal}
\begin{enumerate}
    \item O administrador seleciona a turma desejada.
    \item Adiciona ou remove alunos da turma.
    \item Associa ou remove professores da turma.
    \item O sistema atualiza os relacionamentos.
\end{enumerate}

\subsubsection*{Pós-condição}
\begin{itemize}
    \item A composição da turma é atualizada.
\end{itemize}

\subsubsection*{Fluxo Alternativo}
\begin{itemize}
    \item O sistema impede alterações caso haja pendências (ex.: notas não lançadas).
\end{itemize}

%------------------------------------------------------
\subsection{Registrar e Gerenciar Notas}

\subsubsection*{Atores}
\begin{itemize}
    \item Professor
\end{itemize}

\subsubsection*{Pré-condição}
\begin{itemize}
    \item Professor logado.
    \item Turma, Aluno e Avaliação previamente cadastrados.
\end{itemize}

\subsubsection*{Fluxo Principal}
\begin{enumerate}
    \item O professor seleciona a turma.
    \item Seleciona ou cria a avaliação.
    \item Insere notas (float) para cada aluno.
    \item O sistema armazena as notas.
\end{enumerate}

\subsubsection*{Pós-condição}
\begin{itemize}
    \item As notas da avaliação são registradas.
\end{itemize}

\subsubsection*{Fluxo Alternativo}
\begin{itemize}
    \item Notas inválidas resultam em erro e solicitação de correção.
\end{itemize}

%------------------------------------------------------
\subsection{Registrar Incidente de Comportamento}

\subsubsection*{Atores}
\begin{itemize}
    \item Professor
\end{itemize}

\subsubsection*{Pré-condição}
\begin{itemize}
    \item Professor logado.
    \item Aluno selecionado.
\end{itemize}

\subsubsection*{Fluxo Principal}
\begin{enumerate}
    \item O professor abre a função de Registro de Comportamento.
    \item Seleciona o aluno.
    \item Preenche tag, data e observação.
    \item O sistema cria o registro de comportamento.
\end{enumerate}

\subsubsection*{Pós-condição}
\begin{itemize}
    \item O incidente é armazenado e associado ao aluno.
\end{itemize}

\subsubsection*{Fluxo Alternativo}
\begin{itemize}
    \item Tentativa inválida gera mensagem de erro.
\end{itemize}

%------------------------------------------------------
\subsection{Calcular Médias e Estatísticas}

\subsubsection*{Atores}
\begin{itemize}
    \item Professor
\end{itemize}

\subsubsection*{Pré-condição}
\begin{itemize}
    \item Existem notas registradas.
\end{itemize}

\subsubsection*{Fluxo Principal}
\begin{enumerate}
    \item O professor acessa a tela de estatísticas.
    \item O sistema calcula as médias automaticamente.
    \item O sistema exibe os resultados.
\end{enumerate}

\subsubsection*{Pós-condição}
\begin{itemize}
    \item Estatísticas exibidas com sucesso.
\end{itemize}

\subsubsection*{Fluxo Alternativo}
\begin{itemize}
    \item Caso não existam notas, o sistema exibe mensagem de erro.
\end{itemize}

%------------------------------------------------------
\subsection{Analisar Desempenho e Comportamentos}

\subsubsection*{Atores}
\begin{itemize}
    \item Professor
    \item Coordenador
\end{itemize}

\subsubsection*{Pré-condição}
\begin{itemize}
    \item Usuário acessa o painel de análises.
\end{itemize}

\subsubsection*{Fluxo Principal}
\begin{enumerate}
    \item Usuário seleciona filtros (aluno, turma, conteúdo).
    \item O sistema gera gráficos e indicadores.
    \item Usuário revisa as informações exibidas.
\end{enumerate}

\subsubsection*{Pós-condição}
\begin{itemize}
    \item Indicadores são exibidos.
\end{itemize}

\subsubsection*{Fluxo Alternativo}
\begin{itemize}
    \item Usuário pode adicionar novo comportamento durante a análise.
\end{itemize}

%------------------------------------------------------
\subsection{Consultar Histórico Escolar e Desempenho}

\subsubsection*{Atores}
\begin{itemize}
    \item Aluno
\end{itemize}

\subsubsection*{Pré-condição}
\begin{itemize}
    \item Aluno logado.
\end{itemize}

\subsubsection*{Fluxo Principal}
\begin{enumerate}
    \item O aluno solicita a consulta do histórico.
    \item O sistema recupera registros de Avaliação e Comportamento.
    \item O sistema exibe notas, datas e observações.
\end{enumerate}

\subsubsection*{Pós-condição}
\begin{itemize}
    \item Histórico completo é exibido.
\end{itemize}

\subsubsection*{Fluxo Alternativo}
\begin{itemize}
    \item Se não houver registros, exibir “Nenhum registro encontrado”.
\end{itemize}

%------------------------------------------------------
\subsection{UC09 – Gerar Backup Manual}

\subsubsection*{Atores}
\begin{itemize}
    \item Professor
\end{itemize}

\subsubsection*{Pré-condição}
\begin{itemize}
    \item Sistema possui dados armazenados.
\end{itemize}

\subsubsection*{Fluxo Principal}
\begin{enumerate}
    \item O professor abre as configurações.
    \item Seleciona “Gerar Backup”.
    \item O sistema compacta o banco local.
    \item O usuário salva o arquivo.
\end{enumerate}

\subsubsection*{Pós-condição}
\begin{itemize}
    \item Arquivo de backup gerado.
\end{itemize}

\subsubsection*{Fluxo Alternativo}
\begin{itemize}
    \item Backup sem alterações pode gerar arquivo inconsistente.
\end{itemize}

%======================================================
\newpage
\section{Repositório e Protótipo do Sistema}

Esta seção apresenta os artefatos digitais essenciais para o desenvolvimento do Sisteped, incluindo o repositório oficial do código-fonte e o protótipo elaborado no Figma.

%------------------------------------------------------
\subsection{Repositório no GitHub}

O código-fonte do Sisteped está disponível publicamente no GitHub, permitindo versionamento, colaboração e auditoria do desenvolvimento:

\begin{itemize}
    \item \textbf{Repositório oficial:}\\
    \url{https://github.com/r7melo/Sisteped}
\end{itemize}

O repositório contém:
\begin{itemize}
    \item Estrutura completa do projeto;
    \item Código-fonte de desenvolvimento inicial;
    \item Documentações complementares.
\end{itemize}

%------------------------------------------------------
\subsection{Protótipo}

O protótipo de interface do usuário foi desenvolvido no Figma com o objetivo de validar a experiência de uso e definir os fluxos principais antes da implementação. Ele pode ser acessado por meio do link:

\begin{itemize}
    \item \textbf{Protótipo no Figma:}
    \url{https://www.figma.com/design/WE4tHmzitXWEictT3dCRfe/Sisteped?node-id=0-1\&p=f}
\end{itemize}

O protótipo apresenta as principais telas funcionais do sistema, incluindo:

\begin{itemize}
    \item Tela de Login;
    \item Listagem de Turmas;
    \item Listagem de Alunos;
    \item Listagem de Notas;
    \item Telas de Gráficos de Desempenho;
    \item Tela de Relatórios;
    \item Cadastro de Turmas;
    \item Cadastro de Aluno;
    \item Cadastro de Notas;
    \item Geração de Relatórios.
\end{itemize}

Essas telas representam o fluxo essencial do Sisteped e servem como referência visual para orientar o desenvolvimento da interface final do sistema.

%------------------------------------------------------
\newpage
\section{Formação da Equipe}

A equipe de desenvolvimento foi estruturada para cobrir todas as camadas da aplicação, desde a concepção visual até a persistência de dados:

\begin{itemize}
    \item \textbf{Prototipação e Engenharia de Requisitos:}
    \begin{itemize}
        \item Emanoel
        \item Diego
    \end{itemize}

    \item \textbf{Desenvolvimento Front-End:}
    \begin{itemize}
        \item Diego
        \item André (Author)
        \item Klênisson (Apoio Técnico)
    \end{itemize}

    \item \textbf{Desenvolvimento Back-End:}
    \begin{itemize}
        \item Klênisson
        \item Emanoel
        \item Kauan
    \end{itemize}

    \item \textbf{Administração de Banco de Dados (DBA):}
    \begin{itemize}
        \item Romário
    \end{itemize}
\end{itemize}


\end{document}

